\textit{Note: In general, all integer IDs are auto-incremented.}

\sffamily

\section{School-relevant}
school(\underline{id}: \texttt{INT UNSIGNED},
name: \texttt{VARCHAR(127)})
\\

prefix(\underline{code}: \texttt{VARCHAR(2)},
school\_id*: \texttt{INT UNSIGNED})
\\

programme\_school(\underline{school\_id}*: \texttt{INT UNSIGNED},
\underline{school\_id}*: \texttt{INT UNSIGNED})
\\

\textit{Note: Modules keep track of their offering school.}

\section{Programme-relevant}
programme(\underline{id}: \texttt{INT UNSIGNED},
name: \texttt{VARCHAR(127)},
duration: \texttt{VARCHAR(63)},
\`{}type\`{}: \texttt{VARCHAR(15)})
\\

requirement(\underline{id}: \texttt{INT UNSIGNED},
min\_year\_of\_study: \texttt{TINYINT(1) UNSIGNED},\\
max\_year\_of\_study: \texttt{TINYINT(1) UNSIGNED},
programme\_id*: \texttt{INT UNSIGNED})
\\

\`{}group\`{}(\underline{id}: \texttt{INT UNSIGNED},
min\_credits: \texttt{INT(3) UNSIGNED},
max\_credits: \texttt{INT(3) UNSIGNED},
min\_grade: \texttt{TINYINT(2) UNSIGNED},
max\_grade: \texttt{TINYINT(2) UNSIGNED},
requirement\_id*: \texttt{INT UNSIGNED})
\\

module\_group(\underline{group\_id}*: \texttt{INT UNSIGNED},
\underline{module\_code}*: \texttt{VARCHAR(6)})
\\

\textit{Note: There are no tables for the relationships, as requirement and
\`{}group\`{} keep track of what programme or requirement they belong to
respectively; A module group is a disjunction of modules as part of a
requirement (and should probably be renamed, along with \`{}group\`{}).}

\section{Module-relevant}
academic\_year(\underline{title}: \texttt{VARCHAR(9)})
\\

module(\underline{code}: \texttt{VARCHAR(6)},
name: \texttt{VARCHAR(255)},
description: \texttt{VARCHAR(2047)},
credit\_worth: \texttt{INT(3) UNSIGNED},
re\_assessable: \texttt{BOOL},
external\_requirement: \texttt{TEXT},
school\_id*: \texttt{INT UNSIGNED})
\\

graded\_through(\underline{module\_code}*: \texttt{VARCHAR(6)},
\underline{assessment\_type}*: \texttt{VARCHAR(63)},
\underline{academic\_year}*: \texttt{VARCHAR(9)},
percentage: \texttt{TINYINT(3) UNSIGNED})
\\

assessment(\underline{\`{}type\`{}}: \texttt{VARCHAR(63)},
description: \texttt{TINYTEXT})
\\

timetable(\underline{module\_code}*: \texttt{VARCHAR(6)},
\underline{academic\_year}*: \texttt{VARCHAR(9)},\\
\underline{url}: \texttt{VARCHAR(255)})
\\

time\_frame(\underline{module\_code}*: \texttt{VARCHAR(6)},
\underline{\`{}start\`{}}: \texttt{DATE},
\underline{\`{}end\`{}}: \texttt{DATE})

    \subsection{Requisite-relevant}
    requisite(\underline{id}: \texttt{INT UNSIGNED},
    academic\_year*: \texttt{VARCHAR(9)},\\
    semester\_number*: \texttt{TINYINT(1) UNSIGNED},
    academic\_level: \texttt{VARCHAR(3)},\\
    \`{}type\`{}: \texttt{VARCHAR(4)},
    source\_module*: \texttt{VARCHAR(6)})
    \\
    
    requisite\_group(\underline{group\_id}: \texttt{INT UNSIGNED},
    requisite\_id*: \texttt{INT UNSIGNED})
    \\
    
    disjunctive\_group(\underline{group\_id}*: \texttt{INT UNSIGNED},
    \underline{module\_code}*: \texttt{VARCHAR(6)})
    \\
    
    \textit{Note: Requisites keep track of the module they belong to; Requisite
    groups keep track of the requisite they belong to and exist purely to
    provide a requisite group ID.}

    \pagebreak

    \subsection{Teaching-relevant}
    taught\_by(\underline{module\_code}*: \texttt{VARCHAR(6)},
    \underline{lecturer\_uname}*: \texttt{VARCHAR(127)},\\
    \underline{academic\_year}*: \texttt{VARCHAR(9)},
    \underline{semester\_number}*: \texttt{TINYINT(1) UNSIGNED})
    \\
    
    lecturer(\underline{user\_name}: \texttt{VARCHAR(127)},
    title: \texttt{VARCHAR(15)},
    first\_name: \texttt{VARCHAR(127)},
    surname: \texttt{VARCHAR(127)})
    \\
    
    taught\_in(\underline{module\_code}*: \texttt{VARCHAR(6)},
    \underline{semester\_number}*: \texttt{TINYINT(1) UNSIGNED},
    \underline{academic\_year}*: \texttt{VARCHAR(9)})
    \\
    
    semester(\underline{\`{}number\`{}}: \texttt{TINYINT(1) UNSIGNED},
    name: \texttt{VARCHAR(10)},\\
    start\_month: \texttt{int(2) UNSIGNED},
    end\_month: \texttt{INT(2) UNSIGNED})
    \\
    
    taught\_through(\underline{module\_code}*: \texttt{VARCHAR(6)},
    \underline{teaching\_type}*: \texttt{VARCHAR(63)},\\
    \underline{academic\_year}*: \texttt{VARCHAR(9)},
    hours: \texttt{SMALLINT(3) UNSIGNED})
    \\
    
    teaching(\underline{\`{}type\`{}}: \texttt{VARCHAR(63)},
    description: \texttt{VARCHAR(511)})

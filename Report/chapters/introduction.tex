The aim of this project was to facilitate navigation and understanding of the
University of St Andrews undergraduate course catalogue. In its current form,
the course catalogue is a collection of PDF-documents split up by school and by
sub-honours and honours level. This makes it difficult for students and advisers
alike to figure out what modules students should be taking. Further complicating
the matter is that the list of tables in the PDFs give no indication of what
consequences module choices will have. For example not taking the 2000-level
networking module in Computer Science leads to not being able to take CS3102,
which further leads to not being able to take CS4103, CS4105, CS4302, and
CS5022, due to the ``chaining'' of pre-requisites. This consequence of not
taking a second-year, sub-honours module is cumbersome to figure out based on
the PDFs, as one would need to look through all the honours modules in second
year, well beyond the modules relevant for that year or even the next one, which
are listed in a separate document from the one the student would be looking at,
i.e. the sub-honours document(s).

With joint honours, the difficulty increases even more, due to the large variety
of required modules for the programme in combination with the requirements of
other modules the student might be interested in.
\\

This project aims to solve these issues, or at least facilitate the planning
process, by visualising the modules and their requisites as a network of nodes
and edges respectively.

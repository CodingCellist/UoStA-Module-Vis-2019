A database model and implementation capturing module requisites and programme
requirements was achieved. The model has proven to be adequate, as it was able
to store the data scraped from the official university website. Accessing the
database through a server, several visualisations were designed. These have
provided several new insights into the state and structure of requisites, e.g.
various old modules which are still needed as they could have been taken by
students in their later years, and the fragmentation of some sub-categories of
modules. The column-oriented layout has been incorporated into a website which
allows the user to interact with it and filter the data, bringing the system
closer to a deployable state. The column-oriented layout greatly facilitates
establishing an overview and exploring consequences since each level is
immediately visible compared to having to scroll through multiple PDFs. Whilst
the project is not complete, it is now in a stage where someone could further
develop it.
\\

Several additions could be made to the project. The most essential would be
scraping and inserting the degree programmes into the database, as well as
scraping more modules, e.g. from the school of Mathematics and Statistics. Using
the programme data, a system which displayed what programmes were available
given the current module selection could be implemented. Or vice-versa: a system
which highlighted the core modules for selected programme(s) could also be
implemented. Additional filters could be added to filter by programmes, an
academic year range, or perhaps a maximum number of years from the current to
display modules in. Finally, a third network overview called ``hive plots''
should be implemented to provide another overview of the network. My initial
idea was to have the schools be axes and to have the 1000-level modules nearest
the centre, increasing the level going outwards. A potential problem with this
visualisation is the number of requisites between modules in the same school.
However, this visualisation should still be attempted.

Adding a feature to save the selection for retrieval after the current browser
session is beyond the scope of this project, as it would then start to resemble
a competitor to the existing advising system, which is not the aim of the
project.
